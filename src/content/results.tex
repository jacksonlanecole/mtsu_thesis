The primary objective in carrying out this project was to build an
infrastructure to support the investigation into methods of
optimization of initial simulation parameters based on the output of the
JSPAM simulations.
We have accomplished this objective by establishing a working framework with a
elementary approach at comparison of simulation images and real SDSS DR7, DR8,
and DR9 images prepared with MergerEx.

There now exists a dedicated
\texttt{merger} class with member functions that implement most necessary tasks
relating to the merger data. Rather than deal with data I/O on a case by case
basis, we can now instantiate this class to encapsulate all necessary data for
operations common to analysis of mergers in the context with which we are
concerned. Further, the output from all member functions has been logically
structured to reduce future headaches that come from handling large volumes
of file I/O. There also exist several member functions that aid in visualization
of a particular simulation from start to finish.

The \texttt{merger} class is used extensively in what we dub the JSPAM Command
Line Interface (\texttt{jspamcli.py}). This program makes interacting with
the JSPAM code a bit more intuitive and gives the user several modes of
interaction that we found to be useful in different contexts. Further, we have
now standardized I/O from JSPAM so that future work on the project can rely on
predictable I/O behavior.

We also created an option to execute \texttt{basic\_run} asynchronously across
a variable number of cores. While we have not strictly parallelized
\texttt{basic\_run}, we've distributed its execution to make more efficient use
of powerful workstations that are readily available.
