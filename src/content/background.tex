The primary question regarding the observed disturbed morphological features
of interacting galaxy pairs,
described as \say{bridges and tails}, related to the mechanism by which they
were formed.
In 1972, \citet{Toomre1972} presented some of the first widely
accepted supporting
evidence of what they tellingly referred to as \say{old-fashioned gravity}
as a basis for the disturbed morphologies seen in apparently-neighboring
galaxies. \citet{German} provided the
foundation for their work, but their results were largely rejected in the 1950s
and 1960s when the complexities of the bridges and tails were thought to
result from equally complex mechanisms.
During this period, astrophysicists were in some cases vehemently
opposed to using gravitational interactions as the basis of explanation of
disturbed morpholoigcal features. However, it was then
shown using restricted three-body simulations of colliding galaxies
that the supposed static nature of transient tidal forces between
members of galaxy pairs matters less when considering the intensity
of those tidal forces \cite{Toomre1972}.


\citet{Toomre1972} focuses on four simple examples of interacting systems, each
of which serve to illustrate their proposal that the main contributors to the
bridges and tails are kinematic and can be described simply in terms of gravity.
They purposely initialize slow, parabolic encounters between members of the
galaxy pair, as they assume that faster encounters result in thinner tidal
features. This immediately seems like an ad hoc initialization, in which
one \textit{should} find cause for worry that the result may have been biased
towards the desired result. However, they justify their logic by making the
following assumptions. Numerous observable encounters of galaxies traveling along
hyperbolic orbits seems highly improbable, and accordingly, observational data
of the interacting pairs point to the interaction \textit{not} having been a
chance encounter. It seems much more probable that the interacting pair were
already bound, and while still highly eccentric, their approach orbits were
still sub-parabolic. In their work, they initialize a primary disk that
is composed of a point mass at the center, and several concentric rings of
test particles surrounding. The secondary \say{disk} is simply a point mass
representing the center of the secondary galaxy. All numerical integrations were
performed separately for each particle in the primary disk, such that the
computational time could be optimized for \textit{each particle} in the system
\cite{Toomre1972}.

In modern times, the primary challenges involved in simulating
interactions between galaxies are not very far removed from those of
computational economy which were highly necessary considerations in the work
of the brothers Toomre in the early 1970s, although we now certainly perform
the necessary integration for \textit{each} particle at \textit{each} time step.
In many cases, simulations of galactic collisions still make use of restricted
three-body codes similar to those used in the original work in the field.
Restricted three-body codes
only require that the gravitational force between each of the two most massive
bodies in the system and a particular test mass to be calculated. These work
when the gravitational interactions between each smaller (essentially massless)
test mass in the system is assumed to be negligible in comparison to those
involving the larger masses. The primary benefit is that they are significantly
more computationally economical than their $n$-body counterpart.
These are also considered valid, as the self gravity of the galactic disks is
assumed to
be almost negligent in comparison to the gravitational interaction of the two
galactic centers and each smaller mass \cite{Toomre1972}. Of course,
computational economy still is a factor, but its role in guiding the methods
for simulations has changed.

Today, the computational power available to researchers
exceeds by far what was available in the last six decades. Whereas in decades
past, restricted three-body codes were used \textit{primarily} on the basis of
computational economy, we can now use these same types of codes on the basis
of computational economy \textit{when our solution space still contains
\say{bad} solutions.} Essentially, we can now use the faster three-body
algorithms
to accurately simulate significant kinematic interactions (e.g. those between
each point mass and the centers of each galactic disk), and then use either
fitness-functions,
automatic image recognition and classification via machine learning techniques,
Citizen Science, or some combination of the three to constrain the solution
space to only \say{good} or \say{okay} solutions. From our \say{good} solutions,
we can begin to use full $n$-body simulations for further analysis.

It should be reiterated that in comparison to restricted three-body codes,
full $n$-body simulations are naturally computationally expensive.
However, simulations of dynamical processes within the disks of galaxies
are better served by the computational complexity of $n$-body codes than are
purely kinematic processes under the assumptions given by
\citet{Toomre1972}.
If the initial goal of a particular simulation can be oriented
toward finding a valid solution for the initial conditions that contribute to
the observed morphological features,
restricted three-body codes, such as JSPAM \cite{Wallin2016}, serve well
to lessen the computational overhead in running large numbers of simulations
with reasonable resolution.
Rather than immediately deal
with the computationally expensive approach of calculating the self gravity
within the disk material (among the stars) using $n$-body codes,
researchers can use restricted three-body codes to run significantly large
numbers of simulations
relatively quickly, which allows simulation of solutions across the
\textit{entire} solution space.
%Restricted three-body codes, and specifically JSPAM will be discussed further in
%\ref{sec: methods}.\ref{ssec: jspam}.

In short, we want to make efficient use of computational time
and to reduce the length of time the human-in-the-loop must spend
\say{in the loop.}
In recent years, the Galaxy Zoo project made use of what essentially was a
human fitness-function comprised of Citizen Scientists to review
over three million simulations \cite{Holincheck2015}.
However, as of writing of \citet{Holincheck2015}, there existed no general
purpose machine fitness-function for use in determining the level of
convergence on observed morphologies of interacting galaxies. This work aims
primarily to develop a framework for testing general purpose machine
fitness-functions.
The simulated data come from the JSPAM code, found at
\texttt{https://github.com/JSPAM-Manga/WallinCode}; this is a restricted
three-body code originally written in FORTRAN that is described in
\citet{Wallin2016}.

The working fork of the project is currently in regular development and can be
found at \texttt{https://github.com/jacksonlanecole/WallinCode}.
