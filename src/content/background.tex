The primary question regarding observed disturbed morphological features
of interacting galaxy pairs, described as \say{bridges and tails}, relates to
the mechanism by which they were formed.
In 1972, \citet{Toomre1972} presented some of the first widely
accepted supporting
evidence of what they tellingly referred to as \say{old-fashioned gravity}
as a basis for the disturbed morphologies observed in apparently-neighboring
galaxies. \citet{German} provided the
foundation for their work, but their results were largely rejected in the 1950s
and 1960s when the complexities of galactic bridges and tails were assumed to
result from equally complex mechanisms.
During this period, astrophysicists were in some cases vehemently
opposed to using gravitational interactions as the basis of explanation of
disturbed morphological features. However, it was then
shown using restricted three-body simulations of colliding galaxies
that the supposed static nature of transient tidal forces between
members of galaxy pairs matters less when considering the magnitude of the
intensity of those tidal forces~\cite{Toomre1972}.


\citet{Toomre1972} focuses on four simple examples of interacting systems, each
of which serve to illustrate their proposal that the main contributors to the
bridges and tails are kinematic and can be described simply in terms of
gravity.
In each example, they purposely initialize slow,
parabolic encounters between members of the
galaxy pair, as they assume that faster encounters result in thinner tidal
features, and the primary goals of their work focused on proving that gravity
could be one of the sole causes of the observed features.
This immediately seems like an ad hoc initialization, in which
one should find cause for worry that the result may have been biased
towards the desired result. However, they justify their logic by making several
assumptions.

A relatively high frequency of observable encounters of
galaxies traveling along chance, unbound, hyperbolic orbits is
highly improbable, and accordingly,
observational data from many interacting pairs point to their interactions
not having been chance encounters.
It is much more probable that a significant number of interacting pairs were
already bound, and while still highly eccentric, their approach orbits were
still sub-parabolic, meaning that their orbits could readily become bound.
In work the work of \citet{Toomre1972}, a primary disk is initialized that
is composed of a point mass at the center that is surrounded by several
concentric rings of massless test particles.
The secondary \say{disk} is simply a point mass
representing the center of the secondary galaxy. All numerical integrations were
performed separately for each particle in the primary disk, such that the
computational time could be optimized for each particle in the
system~\cite{Toomre1972}. From the work of~\citet{Toomre1972}, it becomes clear
that gravity alone could be a fundamental cause of the observed morphological
features.

In modern times, the primary challenges involved in simulating
interactions between galaxies are not very far removed from those of
computational economy which were highly necessary considerations in the work
of the brothers Toomre in the early 1970s, although we now at least perform
the necessary integration for each particle at each time step.
In many cases, simulations of galactic collisions still make use of restricted
three-body codes similar to those used in the original work in the field.
Restricted three-body codes
only require that the gravitational force between each of the two most massive
bodies in the system and a particular test mass is calculated. These codes work
when the gravitational interactions between each smaller (essentially massless)
test mass in the system is assumed to be negligible in comparison to those
involving the larger masses.
Their primary advantage over their $n$-body counterpart is their computational
economy.
These codes are also still considered valid, as the self gravity of the galactic
disks is assumed to be almost negligible in comparison to the gravitational
interaction of the two galactic centers and each smaller mass~\cite{Toomre1972}.
Of course, computational economy still is a factor, but its role in guiding
the methods for simulations has changed.

Today, the computational power available to researchers
exceeds by far what has been available in the last six decades.
Whereas in decades past, restricted three-body codes were used
primarily on the basis of computational economy, we can now use these same
types of codes as a still-valid method to quickly reduce the size of the
solution space while the space is still populated with
obviously poor solutions.
Essentially, faster three-body algorithms can be used
to accurately simulate more significant kinematic interactions
(e.g.\ those between each point mass and the centers of each galactic disk),
and then fitness-functions,
automatic image recognition and classification via machine learning techniques,
Citizen Science, or some combination of the three can be used to constrain
the solution space to solutions that have a high likelihood of having
reasonable convergence on the actual solution. From the constrained solution
space, full $n$-body simulations can be used for further analysis.

The computational expense of full $n$-body codes is justified by
when simulating the dynamical processes within the disks of galaxies, as these
effects can be accurately simulated for each particle.
If restricted three-body codes, such as JSPAM \cite{Wallin2016}, can be
used to accurately simulate the large-scale kinematic interactions with
reasonable resolution, then they can be used to accurately simulate solutions
across the entire solution space fairly quickly.
From that point, these computationally expensive codes can be used to explore
the reduced solutions space.

In short, we want to make efficient use of computational time
and to reduce the size but increase the complexity of the job left for the
human-in-the-loop.
In recent years, the Galaxy Zoo project used what essentially was a
human fitness-function comprised of Citizen Scientists to review
over three million simulations~\cite{Holincheck2015}.
However, as of writing of~\citet{Holincheck2015}, there existed no general
purpose machine fitness-function for use in determining the level of
convergence of solutions on observed morphologies of interacting galaxies.
This work does not provide a machine fitness-function, but provides tools that
are useful in support of this effort.

This work primarily focuses on developing methods for interacting with JSPAM,
a restricted three-body code originally written in FORTRAN that is described in
\citet{Wallin2016} that can be found at
\texttt{https://github.com/JSPAM-Manga/WallinCode}.
JSPAM simulates these interactions and produces data that can be used to render
the simulated morphologies for eventual use in machine fitness-functions.
However, a standard procedure and pipeline for operating the
JSPAM code and handling the resulting data needed to be developed.
