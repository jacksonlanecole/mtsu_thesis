While our goal of establishing an interface for using and
interacting with data from the JSPAM code has been achieved,
it is clear that this work really lies mostly in a support role of other
projects. That being said, successful data-intensive projects rely heavily on
user-friendly data pipelines that handle most of the tedious labor and keep
track of all relevant data in a predictable, well thought-out manner.
We expect that the framework
we have created will be useful in that it lessens the overhead necessary to
bootstrap a project using the JSPAM code for its simulation software by
eliminating the need to create ad-hoc methods for interacting with the
simulation data. At this point, users are instructed on how to best interact
with the software for their needs using \texttt{jspamcli.py}, and there is a
clear method by which the simulation output data is stored and organized.

Because this framework is now well established, it becomes trivial to set
up a batch
run file to produce \textit{all} 66,395 sets of run data and then the
associated rendered images.
Essentially, this gives us a massive set of ranked simulation data that
can serve as a training set for future attempts at improving the results of
JSPAM by applying machine learning techniques.

We recognize near the time of completion of this phase of the project that it
would be quite useful to have a JSPAM interface object that can be used
programmatically and that accepts an initial run string as an argument to a
\say{run} method. In tasks where these initial condition guesses are being
optimized, having an implementation of this would be essential for rapid
testing.

While this project has not arrived at scientific results, we intend for our
code to handle some of the
heavy lifting in future projects electing to use the JSPAM code.
