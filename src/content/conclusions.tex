While our goal of establishing a framework for testing various fitness functions
and interacting with the JSPAM code has been achieved, it is clear that the
actual fitness function used is decidedly \textit{not} the ideal fitness
function to use in the future. That being said, we expect that the framework
we have created will be useful in that it lessens the overhead necessary to
bootstrap a project using the JSPAM code for its simulation software by
eliminating the need to create ad-hoc methods for interacting with the
simulation data. At this point, users are instructed on how to best interact
with the software for their needs using \texttt{jspamcli.py}, and there is a
clear method by which the simulation output data is stored and organized.

Further, users now have available to them in the project repository all
available sets of initial simulation parameters that received a Citizen Science
score, as well as a directory containing real color and calibrated (to the needs
of the comparison code) target photographic images and initial simulation
parameter files all organized by the target name.

Because this framework is now well established, we can easily set up a batch run
file to produce \textit{all} 66,395 sets of run data and then the associated
rendered images. Essentially, this gives us a scored set of simulation data that
can serve as a significant training set on which to train models in future
attempts at applying appropriate machine learning algorithms to quickly reduce
the solution space presented to a human for analysis to the \say{best of the
best.}

While this project has not arrived at scientific results, we intend for our
heavy lifting to aid in all future projects electing to use the JSPAM code.
