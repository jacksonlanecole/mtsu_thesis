We heavily explored several alternative methods for passing
large volumes of data between
the working languages in the project (Fortran90, Python, C++), as writing out
data to flat files carries an intrinsic overhead that could be avoided
altogether.
Binary data
formats such as HDF5 seemed promising initially, but the benefit of having a
small improvement in speed at run-time did not outweigh the cost of having to
write HDF5 handlers in all of the working languages of the project.

We were also well aware that many languages have interfaces for passing
arrays and other common data types between languages (e.g. Python's ctypes
library, or using some implementation of a boost wrapper).
However, for similar reasons to that which kept us from pursuing HDF5, we opted
not to write custom interfaces for passing data. While we would likely have seen
a speed increase at run-time if we were also creating and analyzing images in
the loop, the overhead of writing three separate interfaces was not outweighed
by the marginal gains in efficiency we would have seen.
Therefore, data storage using flat ASCII files are the working paradigm
until other solutions are deemed necessary.

Because we are using flat files, we adopted a standard naming convention for all
output files regardless of their storage location. For any one particular run of
the JSPAM code, we identify a unique output file via the SDSSID or name used,
the run number corresponding to the line number in the input file, a flag
indicating whether the data corresponds to the initial output or the final
output, and the number of particles initialized in each galaxy. The output files
for a unique run would have the form shown in Listing~\ref{listing:uniqueform}.
%
\begin{samepage}
\begin{lstlisting}[caption={Output files from JSPAM}, label={listing:uniqueform}]
name.run_number.i.n1_particles.n2_particles.txt
name.run_number.f.n1_particles.n2_particles.txt
\end{lstlisting}
\end{samepage}
%
Further, we needed to have an appropriate directory tree set up for any
particular run. We can visualize this tree in Figure \ref{fig:the_tree}
%
\begin{figure}[H]
    \dirtree{%
        .1 root.
        .2 output.
        .3 \{SDSSID\}.
        .4 \{SDSSID\}.humanscores.txt.
        .4 run0000.
        .5 \{SDSSID\}.0000.i.\{n1\_particles\}.\{n2\_particles\}.txt.
        .5 \{SDSSID\}.0000.f.\{n1\_particles\}.\{n2\_particles\}.txt.
        .4 run0001.
        .5 \{SDSSID\}.0001.i.\{n1\_particles\}.\{n2\_particles\}.txt.
        .5 \{SDSSID\}.0001.f.\{n1\_particles\}.\{n2\_particles\}.txt.
        .4 run0002.
        .5 \{SDSSID\}.0002.i.\{n1\_particles\}.\{n2\_particles\}.txt.
        .5 \{SDSSID\}.0002.f.\{n1\_particles\}.\{n2\_particles\}.txt.
        .4 run\ldots.
        .5 \{SDSSID\}.\{\ldots\}.i.\{n1\_particles\}.\{n2\_particles\}.txt.
        .5 \{SDSSID\}.\{\ldots\}.f.\{n1\_particles\}.\{n2\_particles\}.txt.
    }
    \caption[Output directory tree]{Output directory tree}
    \label{fig:the_tree}
\end{figure}
%
At this point \texttt{jspamcli.py} only needs minor additions to work with
the rendered image creation software and difference code which comprised the
other half of this project. Incorporating all pieces of the project in an
automated machine learning framework would require only a few augmentations
to the existing code, and would support the ultimate goal of optimizing
the parameter space for better convergence on solutions.
This is likely to be explored in continuing research on this project.
