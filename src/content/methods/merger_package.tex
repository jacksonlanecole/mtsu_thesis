One of the primary challenges that arises in astronomy or astrophysics is
simply doing adequate book-keeping of data and related meta-data.
This project is no different.
It was immediately necessary to develop a small,
purpose-built Python package to do just that. Within the \texttt{merger} package
found in the project repository, we make available the \texttt{Galaxy} and
\texttt{MergerRun} classes, which can be see in Figures \ref{fig:cdgalaxy} and
\ref{fig:cdmergerrun}, respectively.
The \texttt{Galaxy} class only carries information related to the number of
particles instantiated in the galaxy and the real and image positional
information. \texttt{MergerRun} encapsulates all information and data relating
to any particular run of JSPAM given some set of initial parameters. This class
also contains methods for determining storage locations for output data,
renaming files according to the appropriate naming conventions, handling the
scores that are generated from fitness-functions, and even some handy plotting
and visualization tools.
%
\subimport{uml/}{galaxy.tex}
\subimport{uml/}{merger_run.tex}
%
Instantiation of a \texttt{MergerRun} is intended to require the fewest number
of arguments possible to reduce book-keeping. It does so by utilizing an
information file written by MergerEx during image preparation, which contains
a list of pertinent information. One of the member functions within the
\texttt{MergerRun} class creates a dictionary from this file, from which it then
initializes several of the class instance attributes. An instance of this
class also requires as parameters the number of particles to be initialized
in each disk, the run number, and the initial simulation parameters that are
found in the input files. In general, this can be done via the line in Listing
~\ref{listing:mergerruninstant}.
%
\begin{lstlisting}[caption={Instantiating \texttt{MergerRun}}, label={listing:mergerruninstant}]
merger_instance = MergerRun(path_to_info_file, n1_particles, n2_particles, run_number, init_sim_param_string)
\end{lstlisting}
%
Having a dedicated MergerRun class allows for improved portability of
data and information relating to a particular run of a merger.
We expect
programming using this class to remain a bit simpler and perhaps more clear.

The MergerRun class also instantiates twice the previously mentioned
\texttt{Galaxy} class;
one is referred to as \texttt{primary} and the other \texttt{secondary}.
We can therefore more clearly keep track of the SDSS ID or
name and information relating to the disk centers,
the right ascension and declination, and
the Cartesian coordinates relative to the pixels in the frame.

Within this package also a small package called \texttt{data\_tools} containing,
as of writing, one class called \texttt{Structure}. \texttt{Structure} accepts
an ordered list of directory and subdirectory names that will comprise a
directory structure to be created, and further, provides a method that will
check for the prior existence of the structure to prevent overwriting of data.
Although this may seem unnecessary, this reduces the chance of error in
organization of the large volume of simulation data that will be produced. As an
example, the line in Listing~\ref{listing:structureusage}
will create the following path in your current working directory:
\texttt{root\_name/child\_1/child\_2/child\_3/}.
%
\begin{lstlisting}[caption={\texttt{Structure} usage}, label={listing:structureusage}]
dir_structure = Structure(["root_name", "child_1", "child_2", "child_3"])
\end{lstlisting}
%
%Further, we also provide \texttt{get\_target\_data}.
%This is a module that scrapes
%\texttt{https://data.galaxyzoo.org/mergers.html} for all of the
%available target data, allows for a user to select from the targets available,
%downloads and unarchives the data into an \texttt{input} directory in the
%working directory. This is more apparently useful in the interactive
%mode of \texttt{jspamcli.py}.
