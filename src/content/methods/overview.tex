%One of the main goals for the JSPAM project as a whole
%is development of a general purpose fitness-function that would allow
%for the generation of preliminary machine scores for all sets of initial
%simulation parameters present for each of the 62 Galaxy Zoo: Mergers targets
%found at \texttt{https://data.galaxyzoo.org/mergers.html}.

The Galaxy Zoo: Mergers project produced ranked scores that are indicative of
the degree to which the morphologies resulting from different sets of
initial conditions correlate with their real counterparts. Essentially, the
scores should relate to how closely the simulated galaxy merger looks like the
real thing.
These rankings are the product of a substantial Citizen Science effort comprising
the Galaxy Zoo: Mergers project, and at the completion of the project,
66,395 sets of initial parameters had been scored through volunteer review of
over 3 million simulations~\cite{Holincheck2015}.
While this effort was successful, moving towards a general-purpose machine
scoring mechanism for these types of simulations would allow for human effort to
be reserved for classifying only solutions that have a reasonable chance
of being correct.

An ideal mechanism would recognize and remove morphologies
that clearly do not match real life; this in contrast with the
current \textit{modus operandi}, which in some cases results in human effort
being wasted on classifying a \say{clearly bad} solution as incorrect.
If we can task a computer with making classifications that require
less acute discernment, humans in-the-loop can be tasked with
performing classifications that take advantage of the visual facilities of
the human eye.
Therefore, we operate under the assumption that an ideal machine scoring
mechanism should be able to closely recreate the results of the Galaxy Zoo:
Mergers Citizen Science effort, and that human volunteers could improve upon
those results from that point.

In support of these efforts, and to initiate the reduction of the role of the
human-in-the-loop, the primary goal of our work focuses on
developing an interface and framework that can be used to aid users in
interaction with JSPAM, deal with the challenges of data storage through
standard methods, and aid in future work on projects using JSPAM.
Running the simulation software efficiently, storing and organizing the large
volumes of data that are inherent to this project, accessing these data
efficiently, image creation, and image analysis, are all intrinsic tasks in
this project, and we aim to provide methods for doing all of these things
effectively.
We recognize that further work must be done in developing fitness-functions,
but expect that our work will aid further success in doing so and should
greatly reduce the overhead for continuing the
work on these types of projects.

The methods described in this section will closely adhere to the order in which
they appear in Figure~\ref{fig:flowchart}, and will end with a
discussion of \texttt{jspamcli.py}, which handles most of the aforementioned
tasks, in subsection~\ref{subsec:jspamcli}.
\clearpage
\subimport{../figures/}{project_flowchart.tex}
