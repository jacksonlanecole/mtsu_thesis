While ultimately, our primary interest is the observed morphology of some
interacting galaxy pair, we need to identify a significant number of parameters
relating to the sizes and orientations of the objects in the sky and in images,
respectively, as well as the velocities of the primary and secondary disks.
Essentially, we want to have available to us all possible data points describing
the merger as it is observed and the positional data to systematically
locate each object in the produced images for further analysis.

The software we use for target image preparation and for constraining the
initial simulation parameter values is MergerEx, which is described in
\citet{holincheckThesis} and is found at
\texttt{https://github.com/aholinch/MergerEx}. MergerEx greatly simplifies and
speeds up the process of querying various astronomical image servers for
specific sky coordinates, calibrating the images, and estimating initial
parameters that describe the sizes, orientations, and velocities of the
primary and secondary galactic disks in the merger.

Because the primary data in which we are interested is only the morphology,
we can rely on whichever images display the morphological features most clearly.
The MergerEx software allows us to choose the most appropriate image source from
either the NASA/IPAC Extragalactic Database/STScI Digital Sky Survey (NED/DSS),
or Sloan Digital Sky Survey Data Release 7, 8, or 9 (SDSS DR7, DR8, DR9).
For our purposes, neither the wavelengths comprising the images retrieved nor
the source matter as long as the distribution of luminous stars in the system is
clearly seen in the image~\cite{Holincheck2015}.
In some cases,~\citet{Holincheck2015}
describes that these images can even come from non-scientific data sets as long
as the previously mentioned condition is met.

\citet{Holincheck2015} provides a list of 54 SDSS and 8 Hubble Space Telescope
(HST) targets. For each of these targets, the process of estimation
of the disks and subsequently the parameter ranges must be carried out
as described in Appendix C of~\citet{holincheckThesis}.
In the project repository's \texttt{targets} directory,
there now exists a Python script that allows the user to keep track
of their work as they progress through all 62 targets, or through any number of
targets that are defined within that directory's \texttt{all\_targets.txt} file.
