When run from the command line, the JSPAM software accepts a string containing
estimations of initial conditions that result in the observed morphologies.
Files containing the initial condition strings for each target are found at
\url{https://data.galaxyzoo.org/mergers.html}, but in order for a new user to
get started on the project, they must download one, several, or all of the files
containing the thousands of sets of initial conditions. In order to simplify
this process, the module \texttt{get\_target\_data} is included in the
\texttt{merger} package that scrapes
\url{https://data.galaxyzoo.org/mergers.html} for all of the available files to
download, and then subsequently downloads the requested file or files, and
unzips to the correct location. There is also an interactive feature that can be
selected via an argument passed to the function. As of right now, all data are
present in the project repository and were downloaded and organized using this
module.

In many cases during this project, we wanted to work only with
initial conditions that result in simulations that would have received a
Citizen Science score, and therefore elected to remove
sets of initial conditions that resulted in morphologies that were rejected
in the citizen science effort. To do this, a small, ad hoc script was written
that can now be found in the \texttt{archive} directory in the project
repository.
The input data files that contain the initial simulation
parameters are located in the \texttt{input} directory
(\texttt{jspamcli.py} expects this to be their location).
These have been reduced in size from the
original files found at \texttt{https://data.galaxyzoo.org/mergers.html}, as
they now contain only initial simulation parameters from scored runs.
