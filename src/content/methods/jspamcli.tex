Usage of the original JSPAM code described in \citet{Wallin2016}
requires a string argument of initial simulation parameters, and accepts a
number of options to specify run settings and simulation quantities such as
the number of particles to be initialized around each center-of-mass.
While all of the existing functionality has been fundamentally unchanged, we
set out to make interaction with the software more user friendly and to
initiate the minimization of the human-in-the-loop by automating several
tedious necessary actions. The result incorporates the previously mentioned
classes as a command line interface for JSPAM,
\texttt{jspamcli.py},
which now handles most of the overhead associated with running the software.

\texttt{jspamcli.py} can be used in four modes that are summarized
in~\ref{table:jspamcli_table}.
Mode $1$ (\texttt{python jspamcli.py -i}) allows for users to process runs for
single targets interactively. This is the most straightforward usage, and allows
for future users to become familiar with the software.
The user is given an option to download files used for input using
the \texttt{get\_target\_data} module, and then is allowed to select from
available input files to run the program.
The user is then asked to enter the number of particles to be initialized in
each galaxy. Then, the interface proceeds in
essentially the same way as in the batch processing modes, which are here
described.

Before \texttt{jspamcli.py} can be run in any of the batch processing modes
(modes $2\to4$), one or more batch run files must be specified. If none are
specified, the program writes out a sample batch run file named
\texttt{sample.txt} in the \texttt{batch\_run\_file} directory.
The contents of a sample batch run file can be seen in
Listing~\ref{listing:samplebrf}.
%
\begin{lstlisting}[caption={Sample batch run file}, label={listing:samplebrf}]
#target,n1_particles,n2_particles,first_run,last_run
587722984435351614,500,500,100,100
\end{lstlisting}
%
However, if in the last column of a row there is the keyword \texttt{ALL}, the
values given for \texttt{first\_run} and \texttt{last\_run} are ignored
and all available non-rejected scored runs are processed. This file would read
as shown in Listing~\ref{listing:samplebrfa}.
%
\begin{lstlisting}[caption={Batch run file for all runs}, label={listing:samplebrfa}]
#target,n1_particles,n2_particles,first_run,last_run
587722984435351614,500,500,000,000,ALL
\end{lstlisting}
%
Modes $2\to4$ are variations on batch processing modes and all use the
available batch run files.
Mode $2$ (\texttt{python jspamcli.py -bi}) essentially gives the user an
introduction to how batch processing is handled by \texttt{jspamcli.py} and is
useful in contexts where the job to process does not have to be left running
for a exceedingly long duration of time. This mode eventually does a batch
processing job, but walks the user through the steps to successfully create a
batch run file.

Mode $3$ (\texttt{python jspamcli.py -b}) has all of the functionality of
mode $2$ but none of the interactivity.
This mode runs as one would expect and processes all desired runs sequentially.

Mode $4$ (\texttt{python jspamcli.py -bm})
contains methods for distributing the execution all instances of the
\texttt{basic\_run} required to complete the batch processing job across
multiple processes on multiple cores.
We wanted to minimize the need to change the original Fortran90 code, so the
Python \texttt{multiprocessing} module was used to handle distributing calls
to JSPAM to available worker processes, effectively increasing the speed with
which any batch processing job can be completed by a factor of the number of
cores available for use (loosely).
While this does not necessarily reduce the time needed for any one simulation,
as JSPAM has not necessarily been parallelized itself,
this does reduce the amount of time
required to run a large number of simulations fairly quickly just by simply
doing more things at the same time. We can fairly easily significantly
speed up our workflow by making use of the cores available to us on any
workstation or machine.

Within the \texttt{multiprocessing} module is a \texttt{Pool} object that
accepts an argument indicating the number of worker processes to be spawned.
Upon execution of \texttt{jspamcli.py}, the argument passed to the program
indicating the requested number of cores to be used is used to instantiate the
\texttt{Pool} object, although we limit the number of cores to be used to half
of those available to preserve resources and good relations with other users.
Each execution of \texttt{basic\_run} using a
unique set of simulation parameters is considered to be completely
independent of all other processes, as the inputs for each simulation do not
depend on the results of the previous simulations. Therefore, we care very
little if all simulations requested run sequentially. We can therefore use the
\texttt{map\_async} module to distribute calls to the appropriate calling
function across the spawned worker processes.

While we do not need to have any shared memory in order to complete a
\say{distributed} run of \texttt{basic\_run}, we still must consider the fact
that the program writes out to the same directory with each iteration (or at the
end of each run).
Again, the processes are not using a shared memory
location during the run, but they are writing out these files to the same location.
The simple fix was to add an \texttt{-m} flag in \texttt{basic\_run} that
indicates that the program should insert a distinguishing number corresponding
to the process number in the name of the file. This allows for
\texttt{jspamcli.py} to organize the output files appropriately.
%
\begin{table}[h!]
    \centering
    \begin{tabular}{clll}
    \toprule
    Mode & Option       & Behavior                        & Arguments \\
    \midrule
    $1$ & \texttt{-i}  & run interactively               & NONE          \\
    $2$ & \texttt{-bi} & batch process interactively     & NONE          \\
    $3$ & \texttt{-b}  & batch process (normal)%
        & \texttt{batch\_run\_file}\ldots\\
    $4$ & \texttt{-bm} & batch processing on multiple cores & \texttt{num\_cores}
    \texttt{batch\_run\_file}\ldots\\
    $5$ & \texttt{-g}  & GIF Creation Tool               & NONE          \\
    \bottomrule
    \end{tabular}
    \caption[\texttt{jspamcli.py} command line options]%
    {List of command line options
        available in \texttt{jspamcli.py}. In the \texttt{-b} and \texttt{-bm}
        modes, multiple batch run files may be specified as long as they exist in
        the \texttt{batch\_run\_files} directory in the root directory.
    }
    \label{table:jspamcli_table}
\end{table}
%



