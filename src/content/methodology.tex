\subsection{Overview}
\subimport{./methods/}{overview.tex}


\subsection{Image Preparation and Target Data Acquisition}
\subimport{./methods/}{image_prep.tex}

\subsection{Targets and Target Input Files}
\subimport{./methods/}{targets.tex}

\subsection{\texttt{merger} Package}
\subimport{./methods/}{merger_package.tex}

\subsection{Naming and Storage Conventions}
\subimport{./methods/}{naming_and_storage_conventions.tex}

\subsection{JSPAM Command Line Interface (\texttt{jspamcli.py})}\label{subsec:jspamcli}
\subimport{./methods/}{jspamcli.tex}

% ---------------------------------------------------------------------------- %
%\subsection{JSPAM, Restricted three-body Simulations}\label{ssec: jspam}
% ---------------------------------------------------------------------------- %
%\todo[inline]{I was not able to add any review of restricted three-body
%    simulations as of yet. I will be adding more information here, as they are
%    essential for cutting down on computational time when the parameter space
%    is still too wide for any regular, beneficial, simulation of
%non-gravitational physics in full $n$-body simulations.}

%\todo[inline]{For some reason, I did not think to add a description of how the
%    actual simulation is run, as we did not write the simulation. Readers would
%    benefit from having this information included. I will be adding this in the
%future.}


% ---------------------------------------------------------------------------- %
% COMMENTED OUT
% ---------------------------------------------------------------------------- %
\begin{comment}
\subsection{Fitness-function Development}\label{ssec: fitness}
The primary goal of this work as a whole is to make progress on the
development of a fitness-function that will return some kind of machine score.
If we find an efficient method for determining convergence of models on the
observed morphologies that rivals that of a human fitness-function, we can then
begin using this method for real-time analysis of models. Although human
fitness-functions are robust and can make use of our innate pattern-matching
abilities \cite{Holincheck2015}, the ability to determine model convergence more
precisely with improved fitness-functions in
sequence with simulations immediately opens the door to improving the results
of previously applied genetic algorithms and applying new machine learning
techniques for optimizing the parameter space for all possible solutions.

That being said, we would be \textit{wrong} to not acknowledge the innate pattern-recognition abilities of humans that made the success of the Galaxy Zoo
project successful \cite{Holincheck2015}.
Rather than abandoning human interaction altogether once
a successful method for machine scoring is in place, perhaps a better approach
is to allow the machine scoring mechanism remove \say{bad} simulations from the
set that needs to receive a human score, effectively reducing the human choice
from the best of \textit{all} solutions to the best of \textit{good} solutions.

\todo[inline]{The current method of comparison is a simple normalization and
subtraction of translated images.}
\end{comment}
% ---------------------------------------------------------------------------- %
% COMMENTED OUT
% ---------------------------------------------------------------------------- %

% ---------------------------------------------------------------------------- %
%\subsection{Fitness-function Analysis}
% ---------------------------------------------------------------------------- %
%\todo[inline]{
%    This section currently has no information because no work has been done in
%    support of this section. Once we have several fitness functions working,
%    we will begin analyzing. I think since this is the primary goal of the
%    project, it really should live in the ANALYSIS section.
%}

%\subsection{Optimization of Initial Parameter Space}
%\todo[inline]{There will be more information added here in future when
%we arrive at this stage in the project, but I think that this will likely be
%better explored in continued research rather than the thesis.}


